\documentclass{article}

% - style template
\usepackage{base}

% - title, author, etc.
\title{PHYS4000 - Workshop 3}
\author{Tom Ross - 1834 2884}
\date{\today}

% - headers
\pagestyle{fancy}
\fancyhf{}
\rhead{\theauthor}
\chead{}
\lhead{\thetitle}
\rfoot{\thepage}
\cfoot{}
\lfoot{}

% - output files
\newcommand{\wffile}[2]{../output/qho/wf.n_x-#1.#2.txt}
\newcommand{\enfile}[1]{../output/qho/en.#1.txt}
\newcommand{\itfile}[1]{../output/qho/it.#1.txt}

\pgfplotstableread{\enfile{sb}}{\ensb}
\pgfplotstableread{\enfile{nc}}{\ennc}

\pgfplotstableread{\itfile{sb}}{\itsb}
\pgfplotstableread{\itfile{nc}}{\itnc}

% - document
\begin{document}

\tableofcontents

\listoffigures

\listoftables

\clearpage

The entire code repository used to calculate the data for this report, and the
\lilf{tex} file used to produce this \lilf{pdf} document, can be found at
\url{https://github.com/dgsaf/acqm-workshop-3}.

Note that at this point \autoref{sec:dissociative} has not yet been attempted -
it may be at a later date, time permitting.

\section{Quantum Harmonic Oscillator}
\label{sec:qho}

\subsection*{Discussion of Theory}
\label{sec:qho-theory}

\subsection*{Discussion of Implementation}
\label{sec:qho-implementation}

The vibrational wavefunctions were calculated using both the shooting-bisection
and the Numerov-Cooley methods, on a grid
$X_{N} = \lrset{x_{1}, \dotsc, x_{N}}$.
The grid $X_{N}$ was constructed such that $x_{1} = -5$, $x_{N} = 5$, with
varying $N_{k} = 2^{k}$, and hence with correspondingly varying step size
$\lr{\delta x}_{k} = 10 \times 2^{-k}$, for $k = 10, \dotsc 17$.

\subsection*{Wavefunctions}
\label{sec:qho-wavefunctions}

It was observed that the wavefunctions had converged indistinguishably close to
the analytic functions by the first data point $k = 10$, corresponding to
$N = 1024$ and $\lr{\delta x} = 0.0097656$.
Hence, we simply present the wavefunctions for one data point, $k = 15$.

The $n = 0, \dotsc, 3$ wavefunctions, calculated using the shooting-bisection
method with, are compared with the analytic wavefunctions in
\autoref{fig:qho-wf-sb}.

\begin{figure}[h]
  \begin{center}
    \input{wf_sb.tex}
  \end{center}
  \caption[S-B Wavefunctions]{
    The $n = 0, \dotsc, 3$ vibrational wavefunctions, calculated using the
    shooting-bisection method for $k = 15$ (shown in blue), are compared with
    the analytic wavefunctions (shown in red).
    The potential is also presented for clarity (shown in black).
    Note that the calculated wavefunctions have been shifted and scaled to
    $\psi_{i} \to \tfrac{1}{2}\psi_{i} + E_{i}$, while the analytic
    wavefunctions have been shifted and scaled to
    $\psi_{i} \to \tfrac{1}{2}\psi_{i} + E_{i} + 0.05$.
  }
  \label{fig:qho-wf-sb}
\end{figure}

The $n = 0, \dotsc, 3$ wavefunctions, calculated using the Numerov-Cooley
method, are compared with the analytic wavefunctions in \autoref{fig:qho-wf-nc}.

\begin{figure}[h]
  \begin{center}
    \input{wf_nc.tex}
  \end{center}
  \caption[N-C Wavefunctions]{
    The $n = 0, \dotsc, 3$ vibrational wavefunctions, calculated using the
    Numerov-Cooley method for $k = 15$ (shown in blue), are compared with the
    analytic wavefunctions (shown in red).
    The potential is also presented for clarity (shown in black).
    Note that the calculated wavefunctions have been shifted and scaled to
    $\psi_{i} \to \tfrac{1}{2}\psi_{i} + E_{i}$, while the analytic
    wavefunctions have been shifted and scaled to
    $\psi_{i} \to \tfrac{1}{2}\psi_{i} + E_{i} + 0.05$.
    Note also that $n = 1$ calculated wavefunction has a phase shift of $-1$,
    which is an arbitrary artifact of the calculation.
  }
  \label{fig:qho-wf-nc}
\end{figure}

\subsection*{Energies}
\label{sec:qho-energies}

The $n = 0, \dotsc, 3$ vibrational energies, calculated using the
shooting-bisection method, are compared with the analytic energies in
\autoref{tab:qho-en-sb}.

\begin{table}[h]
  \begin{center}
    \pgfplotstabletypeset
    [ multicolumn names
    , col sep=space
    , display columns/0/.style={
      column name={$k = 10$}
      , fixed
      , fixed zerofill
      , dec sep align
      , precision=5
    }
    , display columns/1/.style={
      column name={$k = 11$}
      , fixed
      , fixed zerofill
      , dec sep align
      , precision=5
    }
    , display columns/2/.style={
      column name={$k = 12$}
      , fixed
      , fixed zerofill
      , dec sep align
      , precision=5
    }
    , display columns/3/.style={
      column name={$k = 13$}
      , fixed
      , fixed zerofill
      , dec sep align
      , precision=5
    }
    , display columns/4/.style={
      column name={$k = 14$}
      , fixed
      , fixed zerofill
      , dec sep align
      , precision=5
    }
    , display columns/5/.style={
      column name={$k = 15$}
      , fixed
      , fixed zerofill
      , dec sep align
      , precision=5
    }
    , display columns/6/.style={
      column name={$k = 16$}
      , fixed
      , fixed zerofill
      , dec sep align
      , precision=5
    }
    , display columns/7/.style={
      column name={$k = 17$}
      , fixed
      , fixed zerofill
      , dec sep align
      , precision=5
    }
    , display columns/8/.style={
      column name={Analytic}
      , fixed
      , fixed zerofill
      , dec sep align
      , precision=5
    }
    , every head row/.style={
      before row={\toprule}
      , after row={\midrule}
    }
    , every last row/.style={
      after row=\bottomrule
    }
    ]{\enfile{sb}}
  \end{center}
  \caption[S-B Energies]{
    The $n = 0, \dotsc, 3$ vibrational energies, calculated using the
    shooting-bisection method for $k = 10, \dotsc, 17$, are presented and
    compared with the analytic energies, to 6 significant figures.
    It can be seen that they are already convergent to the analytic energies by
    $k = 12$ - perhaps smaller grid sizes should be investigated further.
  }
  \label{tab:qho-en-sb}
\end{table}

The $n = 0, \dotsc, 3$ vibrational energies, calculated using the
Numerov-Cooley method, are compared with the analytic energies in
\autoref{tab:qho-en-nc}.

\begin{table}[h]
  \begin{center}
    \pgfplotstabletypeset
    [ multicolumn names
    , col sep=space
    , display columns/0/.style={
      column name={$k = 10$}
      , fixed
      , fixed zerofill
      , dec sep align
      , precision=5
    }
    , display columns/1/.style={
      column name={$k = 11$}
      , fixed
      , fixed zerofill
      , dec sep align
      , precision=5
    }
    , display columns/2/.style={
      column name={$k = 12$}
      , fixed
      , fixed zerofill
      , dec sep align
      , precision=5
    }
    , display columns/3/.style={
      column name={$k = 13$}
      , fixed
      , fixed zerofill
      , dec sep align
      , precision=5
    }
    , display columns/4/.style={
      column name={$k = 14$}
      , fixed
      , fixed zerofill
      , dec sep align
      , precision=5
    }
    , display columns/5/.style={
      column name={$k = 15$}
      , fixed
      , fixed zerofill
      , dec sep align
      , precision=5
    }
    , display columns/6/.style={
      column name={$k = 16$}
      , fixed
      , fixed zerofill
      , dec sep align
      , precision=5
    }
    , display columns/7/.style={
      column name={$k = 17$}
      , fixed
      , fixed zerofill
      , dec sep align
      , precision=5
    }
    , display columns/8/.style={
      column name={Analytic}
      , fixed
      , fixed zerofill
      , dec sep align
      , precision=5
    }
    , every head row/.style={
      before row={\toprule}
      , after row={\midrule}
    }
    , every last row/.style={
      after row=\bottomrule
    }
    ]{\enfile{nc}}
  \end{center}
  \caption[N-C Energies]{
    The $n = 0, \dotsc, 3$ vibrational energies, calculated using the
    Numerov-Cooley method for $k = 10, \dotsc, 17$, are presented and
    compared with the analytic energies, to 6 significant figures.
    It can be seen that they are already convergent to the analytic energies by
    $k = 10$ - perhaps smaller grid sizes should be investigated further.
  }
  \label{tab:qho-en-nc}
\end{table}

\subsection*{Iterations}
\label{sec:qho-iterations}

The number of iterations required to calculate the $n = 0, \dotsc, 3$
vibrational energies and wavefunctions using the shooting-bisection method, for
$k = 10, \dotsc, 17$, is presented in \autoref{tab:qho-it-sb}.

\begin{table}[h]
  \begin{center}
    \pgfplotstabletypeset
    [ multicolumn names
    , col sep=space
    , display columns/0/.style={
      column name={$k = 10$}
      , fixed
    }
    , display columns/1/.style={
      column name={$k = 11$}
      , fixed
    }
    , display columns/2/.style={
      column name={$k = 12$}
      , fixed
    }
    , display columns/3/.style={
      column name={$k = 13$}
      , fixed
    }
    , display columns/4/.style={
      column name={$k = 14$}
      , fixed
    }
    , display columns/5/.style={
      column name={$k = 15$}
      , fixed
    }
    , display columns/6/.style={
      column name={$k = 16$}
      , fixed
    }
    , display columns/7/.style={
      column name={$k = 17$}
      , fixed
    }
    , every head row/.style={
      before row={\toprule}
      , after row={\midrule}
    }
    , every last row/.style={
      after row=\bottomrule
    }
    ]{\itfile{sb}}
  \end{center}
  \caption[S-B Iterations]{
    The number of iterations required to calculate the $n = 0, \dotsc, 3$
    vibrational energies and wavefunctions using the shooting-bisection method,
    for $k = 10, \dotsc, 17$, are presented.
  }
  \label{tab:qho-it-sb}
\end{table}

The number of iterations required to calculate the $n = 0, \dotsc, 3$
vibrational energies and wavefunctions using the Numerov-Cooley method, for
$k = 10, \dotsc, 17$, is presented in \autoref{tab:qho-it-nc}.

\begin{table}[h]
  \begin{center}
    \pgfplotstabletypeset
    [ multicolumn names
    , col sep=space
    , display columns/0/.style={
      column name={$k = 10$}
      , fixed
    }
    , display columns/1/.style={
      column name={$k = 11$}
      , fixed
    }
    , display columns/2/.style={
      column name={$k = 12$}
      , fixed
    }
    , display columns/3/.style={
      column name={$k = 13$}
      , fixed
    }
    , display columns/4/.style={
      column name={$k = 14$}
      , fixed
    }
    , display columns/5/.style={
      column name={$k = 15$}
      , fixed
    }
    , display columns/6/.style={
      column name={$k = 16$}
      , fixed
    }
    , display columns/7/.style={
      column name={$k = 17$}
      , fixed
    }
    , every head row/.style={
      before row={\toprule}
      , after row={\midrule}
    }
    , every last row/.style={
      after row=\bottomrule
    }
    ]{\itfile{nc}}
  \end{center}
  \caption[N-C Iterations]{
    The number of iterations required to calculate the $n = 0, \dotsc, 3$
    vibrational energies and wavefunctions using the Numerov-Cooley method,
    for $k = 10, \dotsc, 17$, are presented.
  }
  \label{tab:qho-it-nc}
\end{table}

\clearpage

\section{Dissociative Wave Functions for $\rm{H}_{2}^{+}$}
\label{sec:dissociative}

\begin{figure}[h]
  \begin{center}
    \input{figure_1ssg.tex}
  \end{center}
  \caption[Dissociative Wavefunctions for $1s\sigma_{g}$ PEC]{
    A set of dissociative wavefunctions, for the $1s\sigma_{g}$
    $\rm{H}_{2}^{+}$ potential-energy-curve, (shown in red-to-blue) are
    presented across a range of inter-nuclear distances from
    \SIrange{0}{10}{\bohr}.
  }
  \label{fig:1ssg}
\end{figure}

\begin{figure}[h]
  \begin{center}
    \input{figure_2psu.tex}
  \end{center}
  \caption[Dissociative Wavefunctions for $2p\sigma_{u}$ PEC]{
    A set of dissociative wavefunctions, for the $2p\sigma_{u}$
    $\rm{H}_{2}^{+}$ potential-energy-curve, (shown in red-to-blue) are
    presented across a range of inter-nuclear distances from
    \SIrange{0}{10}{\bohr}.
  }
  \label{fig:2psu}
\end{figure}

\begin{figure}[h]
  \begin{center}
    \input{figure_fc_1ssg.tex}
  \end{center}
  \caption[Kinetic Energy Release Distributions for $1s\sigma_{g}$]{
    The kinetic energy release distributions, for $1s\sigma_{g}$, calculated
    using the Franck-Condon approximation (shown in red-to-blue), are presented
    for the 0, 3, 6, 9 vibrational states of $1s\sigma_{g}$ $\rm{H}_{2}^{+}$,
    across a range of kinetic energy release from \SIrange{0}{30}{\eV}.
    Note that the scaling is very funky due to the scaling of the dissociative
    wavefunctions.
  }
  \label{fig:fc-1ssg}
\end{figure}

\end{document}
