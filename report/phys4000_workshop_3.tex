\documentclass{article}

% - style template
\usepackage{base}

% - title, author, etc.
\title{PHYS4000 - Workshop 3}
\author{Tom Ross - 1834 2884}
\date{\today}

% - headers
\pagestyle{fancy}
\fancyhf{}
\rhead{\theauthor}
\chead{}
\lhead{\thetitle}
\rfoot{\thepage}
\cfoot{}
\lfoot{}

% - output files
\newcommand{\wffile}[2]{../output/qho/wf.n_x-#1.#2.txt}
\newcommand{\enfile}[1]{../output/qho/en.#1.txt}
\newcommand{\itfile}[1]{../output/qho/it.#1.txt}

\pgfplotstableread{\enfile{sb}}{\ensb}
\pgfplotstableread{\enfile{nc}}{\ennc}

\pgfplotstableread{\itfile{sb}}{\itsb}
\pgfplotstableread{\itfile{nc}}{\itnc}

% \def\kpowset{1024,2048,4096,8192,16384,32768,65536,131072}
% \def\kset{10,11,12,13,14,15,16,17}
% \def\kmin{10}
% \def\kmax{17}

\def\kpowset{1024,2048,4096}
\def\kset{10,11,12}
\def\kmin{10}
\def\kmax{12}

\def\nset{0,1,2,3}
\def\nmin{0}
\def\nmax{3}

% - document
\begin{document}

\tableofcontents

\listoffigures

\listoftables

\clearpage

\section{Quantum Harmonic Oscillator}
\label{sec:qho}

\subsection*{Discussion of Theory}
\label{sec:qho-theory}

\subsection*{Discussion of Implementation}
\label{sec:qho-implementation}

The vibrational wavefunctions were calculated using both the shooting-bisection
and the Numerov-Cooley methods, on a grid
$X_{N} = \lrset{x_{1}, \dotsc, x_{N}}$.
The grid $X_{N}$ was constructed such that $x_{1} = -5$, $x_{N} = 5$, with
varying $N_{k} = 2^{k}$, and hence with correspondingly varying step size
$\lr{\delta x}_{k} = 10 \times 2^{-k}$, for $k = 10, \dotsc 17$.

\subsection*{Wavefunctions}
\label{sec:qho-wavefunctions}

It was observed that the wavefunctions had converged indistinguishably close to
the analytic functions by the first data point $k = 10$, corresponding to
$N = 1024$ and $\lr{\delta x} = 0.0097656$.
Hence, we simply present the wavefunctions for one data point, $k = 15$.

The $n = 0, \dotsc, 3$ wavefunctions, calculated using the shooting-bisection
method with, are compared with the analytic wavefunctions in
\autoref{fig:qho-wf-sb}.

\begin{figure}[h]
  \begin{center}
    \input{wf_sb.tex}
  \end{center}
  \caption[Bleh]{
    Blah blah.
  }
  \label{fig:qho-wf-sb}
\end{figure}

The $n = 0, \dotsc, 3$ wavefunctions, calculated using the Numerov-Cooley
method, are compared with the analytic wavefunctions in \autoref{fig:qho-wf-nc}.

\begin{figure}[h]
  \begin{center}
    \input{wf_nc.tex}
  \end{center}
  \caption[Bleh]{
    Blah blah.
  }
  \label{fig:qho-wf-nc}
\end{figure}

\subsection*{Energies}
\label{sec:qho-energies}

\subsection*{Iterations}
\label{sec:qho-iterations}

\end{document}
